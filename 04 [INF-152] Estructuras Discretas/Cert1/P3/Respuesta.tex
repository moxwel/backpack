\documentclass[legalpaper,10pt]{article}
\usepackage[utf8]{inputenc}
\usepackage[activeacute,spanish]{babel}
\usepackage{amsmath, amsfonts, amssymb}
\usepackage{enumerate}
\usepackage{float}
\usepackage{indentfirst}
\usepackage{graphicx}
\usepackage{url}
\usepackage{multicol}
\usepackage{geometry}
\usepackage{fullpage}
\usepackage{fancybox}

\setlength\parindent{0pt}

\begin{document}

\thispagestyle{empty}

\begin{minipage}[t]{0.6\textwidth}

{\LARGE \textbf{INF152} Estructuras Discretas}

{\large Profesores: R. Astudillo -- M. Bugueño}

Universidad Técnica Federico Santa Mar\'{\i}a

Departamento de Informática -- Septiembre 2, 2020.

\end{minipage}
\hfill
\begin{minipage}[t]{0.35\textwidth}
%RELLENE CON SUS DATOS PERSONALES:
\textbf{Nombre}: Maximiliano Sepúlveda \\[0.3cm]
  \textbf{Paralelo}: 200
\end{minipage}

\vspace{0.8cm}

{\Large Certamen 1 -- Pregunta 3}

\vspace{0.4cm}

\textbf{Esta evaluación tiene como máximo 25 puntos del C1}.

\begin{enumerate}[a)]
\item Se definen dos conjuntos \(A,B\) dados por
\begin{gather*}
  A = \{ x \in \mathbb N _0 \mid x\ \text{es impar}, x < 10 \} \\
  B = \{ x \in \mathbb N _0 \mid x\ \text{es primo} \}.
\end{gather*}
Sea \(C\) un conjunto definido por \(C = A \cap B\), enumere los elementos
del conjunto potencia \(2^C\) \textbf{[3 pts]}.

\rule{5cm}{0.4pt}

\underline{\textbf{Respuesta:}}

Considerando que \textbf{0 es par}, y que \textbf{0 y 1 no son primos}, se tienen los siguientes conjuntos:
\begin{gather*}
  A = \{1, 3, 5, 7, 9\} \\
  B = \{2, 3, 5, 7, 11, 13, 17, ...\}
\end{gather*}
Por lo tanto, el conjunto \(C\) esta dado por:
\[ A \cap B = \boxed{C = \{3, 5, 7\}} \]
Entonces, los elementos del conjunto potencia de \(C\) serian:
\[\shadowbox{\(\displaystyle 2^C = \Big\{\quad \varnothing, \quad \{3\}, \quad \{5\}, \quad \{7\}, \quad \{3, 5\}, \quad \{3, 7\}, \quad \{5, 7\}, \quad \{3, 5, 7\} \quad \Big\} \)}\]
En total, \(2^C\) tiene \(2^{|C|}\) elementos, es decir, \(2^3 = \boxed{8}\) . \hfill \(\blacksquare\)



\rule{\hsize}{2pt}

















\item Sean \(A, B\) dos conjuntos arbitrarios y sean \(2^A, 2^B\) sus
respectivos conjuntos potencias. Demuestre que
\[ A \subseteq B \implies 2^A \subseteq 2^B \quad \textbf{[8 pts]} \]
\rule{5cm}{0.4pt}

\underline{\textbf{Respuesta:}}

\textbf{Formalizacion:}

Se definen las siguientes proposiciones para la demostración directa \(p \implies q\):
\begin{itemize}
  \item \(p\) : \(A \subseteq B\)
  \item \(q\) : \(2^A \subseteq 2^B\)
\end{itemize}

\textbf{Demostración:}

Según la definición de subconjuntos:
\[ A \subseteq B \iff \forall x \, (x \in A \implies x \in B) \]
Quiere decir que ``todos los elementos de \(A\) están en \(B\)''. Por lo tanto, se puede concluir que:
\[ \boxed{A \cap B = A} \]
Como todos los elementos de \(A\) están dentro de \(B\), todos los subconjuntos posibles de \(A\) también van a estar dentro de \(B\), es decir:
\[\boxed{2^A \subseteq B}\]
Luego, por propiedades de los conjuntos potencia, estos siempre contienen al conjunto completo como tal, es decir:
\[\boxed{A \subseteq 2^A} \text{ y } \boxed{B \subseteq 2^B}\]
Si juntamos todas las conclusiones, tenemos que:
\[\shadowbox{\(\displaystyle (A \subseteq 2^A) \land (2^A \subseteq B) \land (B \subseteq 2^B) \implies ( A \subseteq 2^A \subseteq B \subseteq 2^B) \)}\]

\textbf{Conclusión:}

Se puede notar como \(2^A\) esta contenido en \(B\), y éste a su vez esta contenido en \(2^B\). Por lo tanto, se puede concluir que:
\[\shadowbox{\(\displaystyle 2^A \subseteq 2^B \)}\]
\(\therefore\) La implicancia es cierta. \hfill \(\blacksquare\)


















\newpage

\item Sea \(f : A \to A\) una función biyectiva, demuestre que \(f \circ f :
  A \to A\) también es biyectiva. \textbf{[12 pts]}

\rule{5cm}{0.4pt}

\underline{\textbf{Respuesta:}}

\textbf{Formalización:}

Se definen las siguientes proposiciones para la demostración directa \(p \implies q\) :

Sea una funcion \(f : A \rightarrow A\)
\begin{itemize}
  \item \(p\) : \(f\) es biyectiva
  \item \(q\) : \(f \circ f\) es biyectiva
\end{itemize}

\textbf{Demostración:}

Si consideramos que \(f\) es biyectiva, significa que es inyectiva y sobreyectiva.

Es decir que, segun la definicion de inyectividad, cumple que:
\begin{enumerate}[\quad 1)]
  \item \(\forall x, y \in A \, \Big((f(x) = f(y)) \implies (x = y)\Big)\)
\end{enumerate}
Si consideramos \(\boxed{x=f(a)}\) e \(\boxed{y=f(b)}\) , entonces se tiene que:
\[\displaystyle \Big(f(f(a)) = f(f(b))\Big) \implies \Big( f(a)=f(b) \Big) \implies ( a = b )\]
Por transitividad de la implicancia, se concluye que:
\[\shadowbox{\(\displaystyle \Big(f(f(a)) = f(f(b))\Big) \implies (a = b)\)}\]
\(\therefore\) Teniendo en cuenta que \(\boxed{(f \circ f)(x) = f(f(x))}\) , entonces \(f \circ f\) es \textbf{inyectiva}.

\rule{5cm}{0.4pt}

Y ahora, si consideramos que es sobreyectiva, segun su definicion, cumple que:
\begin{enumerate}[\quad 2)]
  \item \(\forall y \in A, \exists x \in A \, \Big(y = f(x)\Big)\)
\end{enumerate}
Consideremos entonces que uno de los resultados de \(f\) es \(a \in A\). Si es sobreyectiva, quiere decir que debe existir algun valor \(b \in A\) tal que al evauarlo en \(f\) resulte en \(a\).
\[\boxed{\forall a \in A, \exists b \in A \, \Big( f(b) = a \Big)}\]
Y a su vez, deberia haber un valor \(x \in A\) tal que al evaluarlo en \(f\) resulte en \(b\).
\[\boxed{\exists x \in A \, \Big( f(x) = b \Big)}\]
Juntando las conclusiones, entonces se tiene:
\[\forall a \in A, \, \exists b, x \in A \, \Big( (f(x) = b) \land (f(b) = a) \Big)\]
Como \(b\) ya se define, y existe, entonces se puede reducir a:
\[\shadowbox{\(\displaystyle \forall a \in A, \exists x \in A \, \Big( f(f(x)) = a \Big) \)}\]
\(\therefore\) Teniendo en cuenta que \(\boxed{(f \circ f)(x) = f(f(x))}\) , entonces \(f \circ f\) es \textbf{sobreyectiva}.

\textbf{Conclusión:}

Como se demostró que \(f \circ f\) es inyectiva y sobreyectiva, se puede concluir finalmente que \(f \circ f\) es \\ biyectiva. \hfill \(\blacksquare\)







\end{enumerate}

%RECUERDE PONER NOMBRE, ROL Y PARALELO EN EL ENCABEZADO. ADEMAS, RECUERDE QUE EL NOMBRE DEL ARCHIVO DE ENTREGA DEBERÁ SER SU ROL (ROL.PDF)
\end{document}
