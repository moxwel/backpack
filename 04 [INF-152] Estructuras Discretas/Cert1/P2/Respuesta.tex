\documentclass[legalpaper,10pt]{article}
\usepackage[utf8]{inputenc}
\usepackage[activeacute,spanish]{babel}
\usepackage{amsmath, amsfonts, amssymb}
\usepackage{enumerate}
\usepackage{float}
\usepackage{indentfirst}
\usepackage{graphicx}
\usepackage{url}
\usepackage{multicol}
\usepackage{geometry}
\usepackage{fullpage}
\usepackage{fancybox}

\setlength\parindent{0pt}

\begin{document}

\thispagestyle{empty}

\begin{minipage}[t]{0.6\textwidth}

{\LARGE \textbf{INF152} Estructuras Discretas}

{\large Profesores: R. Astudillo -- M. Bugueño}

Universidad Técnica Federico Santa Mar\'{\i}a

Departamento de Informática -- Septiembre 2, 2020.

\end{minipage}
\hfill
\begin{minipage}[t]{0.35\textwidth}
%RELLENE CON SUS DATOS PERSONALES:
\textbf{Nombre}: Maximiliano Sepúlveda\\[0.3cm]
  \textbf{Paralelo}: 200
\end{minipage}

\vspace{0.8cm}

{\Large Certamen 1 -- Pregunta 2}

\vspace{0.4cm}

\textbf{Esta evaluación tiene como máximo 25 puntos del C1}.



\begin{enumerate}[a)]
\item Sean a, b y c enteros positivos, tales que $a$ divide a $b$ y $b$ divide a $c$. Demuestre que $a$ divide a $c$. \textbf{[8 pts]}

\rule{5cm}{0.4pt}

\underline{\textbf{Respuesta:}}

Primero, cuando se dice que ``un numero \(a\) \textbf{divide} a otro numero \(b\)'', se está diciendo que \(b\) puede representarse como la multiplicación de algún otro numero \(k\) con \(a\); en otras palabras, \(a\) es un factor de \(b\). Con lo anterior, se define entonces:
\[\boxed{\forall a, b \in \mathbb{Z}^+ \, (a \mid b \iff \exists k \in \mathbb{Z}^+ , \, b=a \cdot k)}\]
\[\text{``\(a\) divide a \(b\), si y solo si, existe un numero \(k\) tal que \(b = a \cdot k\)''}\]

\textbf{Formalización:}

A continuación se definen las siguientes proposiciones para la demostración:

Con \(a,b,c \in \mathbb{Z}^+\):
\begin{itemize}
    \item \(p\) : \((a \mid b) \land (b \mid c) \qquad\) \\ ``\(a\) divide a \(b\), y \(b\) divide a \(c\)''

    \item \(q\) : \(a \mid c \qquad\) \\ ``\(a\) divide a \(c\)''
\end{itemize}

\textbf{Demostración:}

*Se utilizará demostración directa. \(p \implies q\)

Supondremos que \(p \equiv \textbf{V}\), y con ello, se va a intentar llegar a \(q\). Por definición, se cumple que:
\begin{enumerate}[\quad 1)]
    \item \(\forall a,b \in \mathbb{Z}^+ \, (a \mid b \iff \exists k \in \mathbb{Z}^+ , \, b = a \cdot k)\)
    \item \(\forall b,c \in \mathbb{Z}^+ \, (b \mid c \iff \exists w \in \mathbb{Z}^+ , \, c = b \cdot w)\)
\end{enumerate}

Y queremos llegar a:
\begin{enumerate}[\quad 3)]
    \item \(\forall a,c \in \mathbb{Z}^+ \, (a \mid c \iff \exists g \in \mathbb{Z}^+ , \, c = a \cdot g)\)
\end{enumerate}

Como se esta considerando \textbf{para cualquier numero} en \(\mathbb{Z}^+\), se puede hacer una particularización universal y considerar \(a,b,c\) números arbitrarios de \(\mathbb{Z}^+\).

Por la proposición 1, se tiene entonces que \(b = a \cdot k\), y en la proposición 2, se tiene que \(c = b \cdot w\). Como en la proposición 2 se encuentra \(b\), se puede reemplazar directamente (ya que lo consideramos como verdadero).
\begin{align*}
    c &= b \cdot w \\
    c &= (a \cdot k) \cdot w \\
    c &= a \cdot (k \cdot w)
\end{align*}

Tenemos en cuenta que también existen los valores de \(k\) y \(w\) en \(\mathbb{Z}^+\), por lo tanto \(k \cdot w\) también existe, y también pertenece a \(\mathbb{Z}^+\), y se le puede definir una variable nueva.
\[c = a \cdot (k \cdot w)\]
\[\shadowbox{\(\displaystyle c = a \cdot g, \qquad g = k \cdot w \)}\]

\textbf{Conclusión:}

Acabamos de encontrar el valor de \(g\) para que la tesis se cumpla. \textbf{Si existe} \(g\) tal que \(c = a \cdot g\). Y además, como son números arbitrarios en \(\mathbb{Z}^+\), se puede generalizar para cualquier numero en \(\mathbb{Z}^+\).

\[\shadowbox{\(\displaystyle \forall a,c \in \mathbb{Z}^+ \, (a \mid c \iff \exists g \in \mathbb{Z}^+ , \, c = a \cdot g) \)}\]

\(\therefore\) Si \(a\) divide a \(b\), y \(b\) divide a \(c\), \textbf{es verdad} que \(a\) también va a dividir a \(c\).
\hfill \(\blacksquare\)























\newpage

\item Demuestre formalmente lo siguiente: \textbf{[8 pts]}
\begin{equation*}
    (a\left[a\left[k\right]\right] = a\left[b\left[j\right]\right]) \rightarrow \exists x \exists y (x=a\left[k\right] \wedge (y=b\left[j\right]) \wedge (a\left[x\right]= a\left[y\right])))
\end{equation*}

\rule{5cm}{0.4pt}

\underline{\textbf{Respuesta:}}

\textbf{Formalización:}

Se consideran las siguientes proposiciones para la demostración: \(p \implies q\)
\begin{itemize}
    \item \(p\) : \(a \Big[ a[k] \Big] = a \Big[ b[j] \Big]\)
    \item \(q\) : \(\exists x \exists y \, \Big((x=a\left[k\right]) \wedge (y=b\left[j\right]) \wedge (a\left[x\right]= a\left[y\right])\Big)\)
\end{itemize}

\textbf{Demostración:}

Si consideramos que \(a[k]\) y \(b[j]\) son elementos específicos de algún conjunto. A la proposición \(p\) se le puede aplicar una generalización existencial:
\[\frac{a \Big[ a[k] \Big] = a \Big[ b[j] \Big]}{\exists x,y \, (a[x] = a[y])}\]

Los elementos que existen para \(x\) e \(y\), son precisamente \(a[k]\) y \(b[j]\) respectivamente.
\[\shadowbox{\(\displaystyle \exists x  \, (x = a[k]) \land \exists y \, (y = b[j]) \land (a[x] = a[y]) \)}\]

\textbf{Conclusión:}

La proposición muestra básicamente como una generalización existencial puede expresar los argumentos de \(a[x]\) en general, con otras variables.

\(\therefore\) Por formalizacion, la expresion \(a \Big[ a[k] \Big] = a \Big[ b[j] \Big]\) tambien se puede expresar como:
\[\shadowbox{\(\displaystyle \exists x \exists y \, \Big((x=a\left[k\right]) \wedge (y=b\left[j\right]) \wedge (a\left[x\right]= a\left[y\right])\Big) \)} \blacksquare\]
















\newpage

\item Demuestre que si $n$ no es un entero divisible por 3, $n^2 - 1 $ sí lo es. \textbf{[9 pts]}

\rule{5cm}{0.4pt}

\underline{\textbf{Respuesta:}}

\textbf{Formalización:}

Se definen las siguientes proposiciones para la demostración \(p \implies q\):

Con \(n \in \mathbb{Z}\):
\begin{itemize}
    \item \(p\) : \(n\) no es divisible por 3.
    \item \(q\) : \(n^2 - 1\) si es divisible por 3.
\end{itemize}

\textbf{Demostración:}

*Se utilizará demostración por contradicción. \((p \land \lnot q) \implies \textbf{F}\)

Supongamos que \(p \equiv \textbf{V}\) y \(\lnot q \equiv \textbf{V}\). Es decir, asumir verdad que:
\begin{enumerate}[\quad 1)]
    \item \(p\) : \(n\) no es divisible por 3.
    \item \(\lnot q\) : \(n^2 - 1\) \textbf{no} es divisible por 3.
\end{enumerate}

De la proposición 1, se sabe que \(n\) no es divisible por 3, es decir, que 3 no divide a \(n\):
\[\boxed{\forall n \in \mathbb{Z} \, \big((3 \nmid n) \iff (\nexists k \in \mathbb{Z} , \, n = 3 \cdot k)\big)}\]

De la proposición 2, de la misma forma, dice que 3 no divide a \(n^2 - 1\):
\[\boxed{\forall n \in \mathbb{Z} \, \big((3 \nmid (n^2 - 1)) \iff (\nexists m \in \mathbb{Z} , \, n^2 - 1 = 3 \cdot m)\big)}\]

Como se considera un \(n \in \mathbb{Z}\) arbitrario, se puede utilizar una particularización universal. Por ejemplo, de la proposición 1, se puede asumir \(\boxed{n = 2}\).

Es valido, ya que no existe ningún numero \(k \in \mathbb{Z}\) tal que al multiplicarlo por 3, me dé 2.

Pero si ahora evaluamos la proposición 2:
\begin{align*}
    n^2 - 1 &= 3 \cdot m \\
    (2)^2 - 1 &= 3 \cdot m \\
    4 - 1 &= 3 \cdot m \\
    3 &= 3 \cdot m
\end{align*}
\[\shadowbox{\(\displaystyle 1 = m \quad \rightarrow\!\leftarrow \quad \nexists m \in \mathbb{Z}\)}\]

\textbf{Conclusión:}

Habíamos supuesto que \(\lnot q\) era verdad. Es decir, que \textbf{no existe} un numero \(m\) tal que \(n^2 - 1 = 3 \cdot m\), pero por la proposición 1 dijimos que \(n = 2\), al final, al unirlo con la proposición 2, nos dio una \textbf{contradicción}.

Es decir, nuestro supuesto inicial acerca de \(\lnot q\) estaba errónea, es decir que \(\lnot q \equiv \textbf{F}\), por lo tanto \(q \equiv \textbf{V}\).

Como se consideraba un \(n \in \mathbb{Z}\) arbitrario, esto se puede generalizar para todo \(n\).

\(\therefore\) Por contradicción, si \(n\) no es divisible por 3, entonces \(n^2 - 1\) si sera divisible por 3. \hfill \(\blacksquare\)

\end{enumerate}

%RECUERDE PONER NOMBRE, ROL Y PARALELO EN EL ENCABEZADO
\end{document}
