\documentclass[letterpaper,10pt]{article}
\usepackage[utf8]{inputenc}
\usepackage[activeacute,spanish]{babel}
\usepackage{amsmath}
\usepackage{amsfonts}
\usepackage{amssymb}
\usepackage{enumerate}
\usepackage{float}
\usepackage{indentfirst}
\usepackage{graphicx}
\usepackage{url}
\usepackage{multicol}
\usepackage{multicol}
\usepackage{subfigure}
\usepackage[position=bottom]{subfig}
\usepackage{geometry}
\usepackage{fullpage}
\usepackage{algorithmic}
\usepackage{algorithm}
\usepackage{fancybox}

\setlength\parindent{0pt}

\usepackage{tikz}
\usetikzlibrary{arrows,petri,topaths,shapes,automata}
\usepackage{tkz-berge}

\tikzset{
    %Define standard arrow tip
    >=stealth',
    % Define arrow style
    pil/.style={
           ->,
           thick,}
}
\begin{document}

\thispagestyle{empty}

\begin{minipage}[t]{0.6\textwidth}

{\LARGE \textbf{INF152} Estructuras Discretas}

{\large Profesores: R. Astudillo -- M. Bugueño}

Universidad Técnica Federico Santa María

Departamento de Informática -- Octubre 30, 2020.

\end{minipage}
\hfill
\begin{minipage}[t]{0.35\textwidth}
%RELLENE CON SUS DATOS PERSONALES:
\textbf{Nombre}: Maximiliano Sepúlveda \\[0.3cm]
\textbf{Rol}: 201973536-5 \textbf{Paralelo}: 200
\end{minipage}

\vspace{0.8cm}

{\Large Certamen 2 -- Pregunta 1}

\vspace{0.4cm}

\textbf{Esta evaluación tiene como máximo 30 puntos del C2}.

\begin{minipage}[b]{0.24\textwidth} %grafo parque
\begin{center}
\begin{tikzpicture}[scale=1.2]
\node (a) at (0.0, 0.0) [circle, scale=0.5, fill=cyan, inner sep = 5pt,label=left:a] {};
\node (b) at (2.0, 0.0) [circle, scale=0.5, fill=cyan, inner sep = 5pt,label=right:b] {};
\node (c) at (0.0, -2.0) [circle, scale=0.5, fill=cyan, inner sep = 5pt,label=left:c] {};
\node (d) at (2.0, -2.0) [circle, scale=0.5, fill=cyan, inner sep = 5pt,label=right:d] {};
\node (e) at (1.5, -0.5) [circle, scale=0.5, fill=cyan, inner sep = 5pt,label=above:e] {};
\node (f) at (0.5,-1) [circle, scale=0.5, fill=cyan, inner sep = 5pt,label=left:f] {};

\path (a) edge (c) (a) edge (b) (a) edge (f) (a) edge (e) (b) edge (d)
(c) edge (d) (d) edge (e) (d) edge (f) (e) edge (f);
\end{tikzpicture}
\captionof{figure}{Grafo G1}
\label{f:grafo1}
\end{center}
\end{minipage}
\hfill
\begin{minipage}[b]{0.24\textwidth} %grafo parque
\begin{center}
\begin{tikzpicture}[scale=1.5]
\node (j) at (0.5, -0.5) [circle, scale=0.5, fill=red, inner sep = 5pt,label=below right:j] {};
\node (c) at (2.0, 0.0) [circle, scale=0.5, fill=red, inner sep = 5pt,label=right:c] {};
\node (k) at (0.0, -2.0) [circle, scale=0.5, fill=red, inner sep = 5pt,label=left:k] {};
\node (o) at (1.5, -1.5) [circle, scale=0.5, fill=red, inner sep = 5pt,label=right:o] {};
\node (r) at (1.5, -0.5) [circle, scale=0.5, fill=red, inner sep = 5pt,label=above:r] {};
\node (t) at (0.5, -1.5) [circle, scale=0.5, fill=red, inner sep = 5pt,label=left:t] {};

\path (j) edge (k) (j) edge (c) (j) edge (t) (j) edge (r) (c) edge (o)
(k) edge (o) (o) edge (r) (o) edge (t) (k) edge [bend left=80] (c);
\end{tikzpicture}
\captionof{figure}{Grafo G2.}
\label{f:grafo2}
\end{center}
\end{minipage}
\hfill
\begin{minipage}[b]{0.24\textwidth} %grafo parque
\begin{center}
\begin{tikzpicture}[scale=1.2]
\node (a) at (0.0, 0.0) [circle, scale=0.3, fill, inner sep = 5pt] {};
\node (b) at (2.0, 0.0) [circle, scale=0.3, fill, inner sep = 5pt] {};
\node (c) at (0, -2) [circle, scale=0.3, fill, inner sep = 5pt] {};
\node (d) at (2, -2) [circle, scale=0.3, fill, inner sep = 5pt] {};
\node (e) at (1,-1) [circle, scale=0.3, fill, inner sep =5pt] {};

\path (a) edge (b) (a) edge (c)
(c) edge (d)
(b) edge (e) (b) edge (d) (a) edge (e) (c) edge (e) (d) edge (e);
\end{tikzpicture}
\captionof{figure}{Grafo G3.}
\label{f:grafo3}
\end{center}
\end{minipage}
\hfill
\begin{minipage}[b]{0.24\textwidth} %grafo parque
\begin{center}
\begin{tikzpicture}[scale=0.5]
\node (b) at (3, 1.5) [circle, scale=0.3, fill, inner sep = 5pt] {};
\node (c) at (0, 3.5) [circle, scale=0.3, fill, inner sep = 5pt] {};
\node (d) at (-3, 1.5) [circle, scale=0.3, fill, inner sep = 5pt] {};
\node (e) at (-2,-2.5) [circle, scale=0.3, fill, inner sep = 5pt] {};
\node (f) at (2,-2.5) [circle, scale=0.3, fill, inner sep = 5pt] {};

\path (b) edge (e) (b) edge (d)
(c) edge (f) (c) edge (e) (d) edge (b) (d) edge (f);
\end{tikzpicture}
\captionof{figure}{Grafo G4.}
\label{f:grafo4}
\end{center}
\end{minipage}

\begin{enumerate}[a)]
\item Considerando los grafos de las Figuras \ref{f:grafo1} y \ref{f:grafo2}, demuestre que ambas arquitecturas son isomorfas. Denote mapeo entre grafos \textbf{a nivel de nodos como a nivel de arcos} justificando debidamente. \textbf{[15 puntos]}

%INDIQUE ACA SU RESPUESTA
\rule{5cm}{0.4pt}

\underline{\textbf{Respuesta:}}

Para demostrar que los grafos $G1$ y $G2$ son isomorfos, es necesario ver las equivalencias de grados de los vértices de cada uno. Para evitar confusiones, se denotara $c_1$ para el vértice $c$ del grafo $G1$, y $c_2$ para el vértice $c$ del grafo $G2$.

A continuación, se revisara el isomorfismo a nivel de vértices, para ello, se va a revisar los grados de los grafos:

\begin{table}[h]
  \centering
  \begin{tabular}{| c | c || c | c |}
    \hline
    $G1$ & Grado & $G2$ & Grado \\
    \hline
    \hline
    $a$   & \textbf{4} & $c_2$ & 3\\
    $b$   & 2 & $k$   & 3\\
    $c_1$ & 2 & $t$   & 2\\
    $d$   & \textbf{4} & $j$   & \textbf{4}\\
    $e$   & 3 & $r$   & 2\\
    $f$   & 3 & $o$   & \textbf{4}\\
    \hline
  \end{tabular}
\end{table}

Se puede notar que si hay una relación entre los grados de cada vértice (en ambos grafos hay 2 vertices de grado 4, hay 2 vertices de grado 3 y 2 vertices de grado 2), por lo tanto, es posible que los grafos si sean isomorfos. A continuación, se va a definir una función $\phi$ que va a \textit{mapear} los vértices de $G1$ a $G2$:
\[\shadowbox{\(\displaystyle \phi (a) = j \qquad \phi (b) = r \qquad\phi (c_1) = t \qquad\phi (d) = o \qquad\phi (e) = c_2 \qquad\phi (f) = k \)}\]
Podemos notar que $\phi$ es biyectiva, por lo tanto, si posee inversa:
\[\shadowbox{\(\displaystyle \phi ^ {-1} (j) = a \quad \phi ^ {-1} (r) = b \quad \phi ^ {-1} (t) = c_1 \quad \phi ^ {-1} (o) = d \quad \phi ^ {-1} (c_2) = e \quad \phi ^ {-1} (k) = f \quad \)}\]

\newpage

Ahora, para ver el isomorfismo a nivel de arcos, puede ser útil revisar su \textbf{matriz de adyacencia}:

\vspace{1em}

\begin{minipage}{0.5\textwidth}
  \begin{center}
    \begin{tabular}{ r | c c c c c c}
      $G1$ & $a$ & $b$ & $c_1$ & $d$ & $e$ & $f$ \\
      \hline
      $a$   &0&\textbf{1}&\textbf{1}&0&\textbf{1}&\textbf{1}\\
      $b$   &\textbf{1}&0&0&\textbf{1}&0&0 \\
      $c_1$ &\textbf{1}&0&0&\textbf{1}&0&0 \\
      $d$   &0&\textbf{1}&\textbf{1}&0&\textbf{1}&\textbf{1} \\
      $e$   &\textbf{1}&0&0&\textbf{1}&0&\textbf{1} \\
      $f$   &\textbf{1}&0&0&\textbf{1}&\textbf{1}&0 \\
    \end{tabular}
  \end{center}
\end{minipage}
\begin{minipage}{0.4\textwidth}
  \begin{center}
    \begin{tabular}{ r | c c c c c c}
      $G2$ & $j$ & $r$ & $t$ & $o$ & $c_2$ & $k$ \\
      \hline
      $j$   &0&\textbf{1}&\textbf{1}&0&\textbf{1}&\textbf{1}\\
      $r$   &\textbf{1}&0&0&\textbf{1}&0&0 \\
      $t$ &\textbf{1}&0&0&\textbf{1}&0&0 \\
      $o$   &0&\textbf{1}&\textbf{1}&0&\textbf{1}&\textbf{1} \\
      $c_2$   &\textbf{1}&0&0&\textbf{1}&0&\textbf{1} \\
      $k$   &\textbf{1}&0&0&\textbf{1}&\textbf{1}&0 \\
    \end{tabular}
  \end{center}
\end{minipage}

\vspace{1em}

Podemos notar a simple vista que ambos poseen los arcos distribuidos de la misma forma.

$\therefore$ Los grafos $G1$ y $G2$ son isomorfos. \hfill \(\blacksquare\)

\vspace{4cm}

\item Considerando los grafos de las Figuras \ref{f:grafo3} y \ref{f:grafo4}, demuestre o refute que los dibujos se pueden realizar sin levantar el lápiz ni doblar la hoja ni repetir trazos. \textbf{[8 puntos]}.

%INDIQUE ACA SU RESPUESTA
\rule{5cm}{0.4pt}

\underline{\textbf{Respuesta:}}

Para que un grafo pueda dibujarse sin levantar el lápiz ni repetir trazos, éste debe poseer al menos un \textbf{camino de Euler} o un \textbf{circuito de Euler}.

Para que un grafo posea un camino de Euler, éste debe poseer \textbf{exactamente dos vertices de grado impar}. Para que sea un circuito de Euler, éste debe poseer \textbf{todos sus vertices con grado par}.

Considerando lo anterior, se puede ver que $G3$ posee mas de 2 vertices con grado impar, y no todos sus vertices son de grado par; es decir, no es ninguna de las dos, no posee ni un camino ni un circuito euleriano.

Ahora si revisamos $G4$, todos sus vertices son de grado par, es decir, que si existe un circuito de Euler, por lo tanto, si se puede dibujar sin levantar el lápiz ni repetir trazos.

$\therefore$ Con las condiciones dadas, $G3$ no se puede dibujar, pero si se puede con $G4$. \hfill \(\blacksquare\)










\newpage


\item Para $n \geq 2$ par, ¿Cuál es la cantidad mínima de arcos que deben quitarse de un grafo completo $K_n$ para obtener exactamente dos componentes conexas con $\frac{n}{2}$ vértices cada una? Escriba dicha cantidad como una función de $n$. \textbf{[7 puntos]}.

%INDIQUE ACA SU RESPUESTA
\rule{5cm}{0.4pt}

\underline{\textbf{Respuesta:}}

Como estamos hablando de grafos completos, las componentes conexas que se pueden formar al eliminar \textbf{la minima cantidad de arcos} también deberían ser grafos completos, pero con la condición de que sea la mitad de vertices. En otras palabras, séase un grafo completo $K_n$, las dos componentes conexas del grafo completo deberían ser dos grafos $K_{n/2}$.


\begin{center}
  \begin{tikzpicture}[scale=0.5]
    \node [] at (-1.3, 1.5) {$K_2=$};

    \node (a) at (0, 3) [circle, scale=0.3, fill, inner sep = 5pt] {};
    \node (b) at (3, 0) [circle, scale=0.3, fill, inner sep = 5pt] {};

    \draw (a) -- (b);
  \end{tikzpicture}
  \hspace{2em}
  \begin{tikzpicture}[scale=0.5]
    \node (a) at (0, 3) [circle, scale=0.3, fill, inner sep = 5pt] {};
    \node (b) at (3, 0) [circle, scale=0.3, fill, inner sep = 5pt] {};

    \draw [color=red, dashed] (a) -- (b);
  \end{tikzpicture}
  \hspace{2em}
  \begin{tikzpicture}[scale=0.5]
    \node (a) at (0, 3) [circle, scale=0.3, fill, inner sep = 5pt] {};
    \node (b) at (3, 0) [circle, scale=0.3, fill, inner sep = 5pt] {};
  \end{tikzpicture}

  Se quitó \textbf{1 arco}. Quedaron 2 grafos disjuntos $K_1$.
\end{center}

\vspace{1em}

\begin{center}
  \begin{tikzpicture}[scale=0.5]
    \node [] at (-1.2, 1.5) {$K_4=$};

    \node (a) at (0, 0) [circle, scale=0.3, fill, inner sep = 5pt] {};
    \node (b) at (3, 0) [circle, scale=0.3, fill, inner sep = 5pt] {};
    \node (c) at (3, 3) [circle, scale=0.3, fill, inner sep = 5pt] {};
    \node (d) at (0, 3) [circle, scale=0.3, fill, inner sep = 5pt] {};

    \draw (a) -- (b);
    \draw (b) -- (c);
    \draw (c) -- (d);
    \draw (d) -- (a);
    \draw (a) -- (c);
    \draw (d) -- (b);
  \end{tikzpicture}
  \hspace{2em}
  \begin{tikzpicture}[scale=0.5]
    \node (a) at (0, 0) [circle, scale=0.3, fill, inner sep = 5pt] {};
    \node (b) at (3, 0) [circle, scale=0.3, fill, inner sep = 5pt] {};
    \node (c) at (3, 3) [circle, scale=0.3, fill, inner sep = 5pt] {};
    \node (d) at (0, 3) [circle, scale=0.3, fill, inner sep = 5pt] {};

    \draw [color=red, dashed] (a) -- (b);
    \draw [color=green] (b) -- (c);
    \draw [color=red, dashed] (c) -- (d);
    \draw [color=green] (d) -- (a);
    \draw [color=red, dashed] (a) -- (c);
    \draw [color=red, dashed] (d) -- (b);
  \end{tikzpicture}
  \hspace{2em}
  \begin{tikzpicture}[scale=0.5]
    \node (a) at (0, 0) [circle, scale=0.3, fill, inner sep = 5pt] {};
    \node (b) at (3, 0) [circle, scale=0.3, fill, inner sep = 5pt] {};
    \node (c) at (3, 3) [circle, scale=0.3, fill, inner sep = 5pt] {};
    \node (d) at (0, 3) [circle, scale=0.3, fill, inner sep = 5pt] {};

    %\draw [color=red] (a) -- (b);
    \draw (b) -- (c);
    %\draw [color=red] (c) -- (d);
    \draw (d) -- (a);
    %\draw [color=red] (a) -- (c);
    %\draw [color=red] (d) -- (b);
  \end{tikzpicture}

  Se quitaron \textbf{4 arcos}. Quedaron 2 grafos disjuntos $K_2$.
\end{center}

\vspace{1em}

\begin{center}
  \begin{tikzpicture}[scale=0.5]
    \node [] at (-3.6, 2) {$K_6=$};

    \node (a) at (1, 0) [circle, scale=0.3, fill, inner sep = 5pt] {};
    \node (b) at (2.3, 2) [circle, scale=0.3, fill, inner sep = 5pt] {};
    \node (c) at (1, 4) [circle, scale=0.3, fill, inner sep = 5pt] {};
    \node (d) at (-1, 4) [circle, scale=0.3, fill, inner sep = 5pt] {};
    \node (e) at (-2.3, 2) [circle, scale=0.3, fill, inner sep = 5pt] {};
    \node (f) at (-1, 0) [circle, scale=0.3, fill, inner sep = 5pt] {};

    \draw (a) -- (b);
    \draw (a) -- (c);
    \draw (a) -- (d);
    \draw (a) -- (e);
    \draw (a) -- (f);

    \draw (b) -- (c);
    \draw (b) -- (d);
    \draw (b) -- (e);
    \draw (b) -- (f);

    \draw (c) -- (d);
    \draw (c) -- (e);
    \draw (c) -- (f);

    \draw (d) -- (e);
    \draw (d) -- (f);

    \draw (e) -- (f);
  \end{tikzpicture}
  \hspace{2em}
  \begin{tikzpicture}[scale=0.5]
    \node (a) at (1, 0) [circle, scale=0.3, fill, inner sep = 5pt] {};
    \node (b) at (2.3, 2) [circle, scale=0.3, fill, inner sep = 5pt] {};
    \node (c) at (1, 4) [circle, scale=0.3, fill, inner sep = 5pt] {};

    \node (d) at (-1, 4) [circle, scale=0.3, fill, inner sep = 5pt] {};
    \node (e) at (-2.3, 2) [circle, scale=0.3, fill, inner sep = 5pt] {};
    \node (f) at (-1, 0) [circle, scale=0.3, fill, inner sep = 5pt] {};

    \draw [color=green] (a) -- (b);
    \draw [color=green](a) -- (c);
    \draw [color=red, dashed] (a) -- (d);
    \draw [color=red, dashed] (a) -- (e);
    \draw [color=red, dashed] (a) -- (f);

    \draw [color=green](b) -- (c);
    \draw [color=red, dashed] (b) -- (d);
    \draw [color=red, dashed] (b) -- (e);
    \draw [color=red, dashed] (b) -- (f);

    \draw [color=red, dashed] (c) -- (d);
    \draw [color=red, dashed] (c) -- (e);
    \draw [color=red, dashed] (c) -- (f);

    \draw [color=green](d) -- (e);
    \draw [color=green](d) -- (f);

    \draw [color=green] (e) -- (f);
  \end{tikzpicture}
  \hspace{2em}
  \begin{tikzpicture}[scale=0.5]
    \node (a) at (1, 0) [circle, scale=0.3, fill, inner sep = 5pt] {};
    \node (b) at (2.3, 2) [circle, scale=0.3, fill, inner sep = 5pt] {};
    \node (c) at (1, 4) [circle, scale=0.3, fill, inner sep = 5pt] {};

    \node (d) at (-1, 4) [circle, scale=0.3, fill, inner sep = 5pt] {};
    \node (e) at (-2.3, 2) [circle, scale=0.3, fill, inner sep = 5pt] {};
    \node (f) at (-1, 0) [circle, scale=0.3, fill, inner sep = 5pt] {};

    \draw (a) -- (b);
    \draw (a) -- (c);
    %\draw [color=red] (a) -- (d);
    %\draw [color=red] (a) -- (e);
    %\draw [color=red] (a) -- (f);

    \draw (b) -- (c);
    %\draw [color=red] (b) -- (d);
    %\draw [color=red] (b) -- (e);
    %\draw [color=red] (b) -- (f);

    %\draw [color=red] (c) -- (d);
    %\draw [color=red] (c) -- (e);
    %\draw [color=red] (c) -- (f);

    \draw (d) -- (e);
    \draw (d) -- (f);

    \draw (e) -- (f);
  \end{tikzpicture}

  Se quitaron \textbf{9 arcos}. Quedaron 2 grafos disjuntos $K_3$.
\end{center}

De esa forma, sucesivamente, se tiene la siguiente tabla de datos:

\begin{table}[h]
  \centering
  \begin{tabular}{| c | c |}
    \hline
    $n$ & $f(n)$ \\
    \hline
    2 & 1 \\
    4 & 4 \\
    6 & 9 \\
    8 & 16 \\
    10 & 25 \\
    $\vdots$ & $\vdots$ \\
    \hline
  \end{tabular}
\end{table}

Se puede notar como la mitad de $n$, al elevarlo al cuadrado da el resultado. Considerando $n$ una cantidad par $\geq 2$ que representa la cantidad de vertices de un grafo completo $K_n$, se tiene que la cantidad de arcos mínimo a eliminar para separar el grafo en dos grafos $K_{n/2}$ esta dada por la función:

\[\shadowbox{\(\displaystyle f(n) = \left( \frac{n}{2} \right)^2 \)}\blacksquare\]





\end{enumerate}

%RECUERDE PONER NOMBRE, ROL Y PARALELO EN EL ENCABEZADO
\end{document}
